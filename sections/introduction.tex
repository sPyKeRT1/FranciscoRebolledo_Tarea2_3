Hace años que la computación es una de las mejores herramientas para la resolución de problemas en diversas áreas debido a que nos permite procesar grandes volúmenes de datos rápidamente, automatizar tareas y encontrar soluciones optimas que manualmente resultaría imposible. Dentro de las ciencias de la computación una de las ramas mas importantes es el \textbf{Análisis y Diseño de Algoritmos}, ya que permite desarrollar métodos eficientes para resolver problemas complejos.\\

Este trabajo se enfoca en comparar estos dos enfoques, considerando su complejidad temporal y espacial, y analizando cómo la naturaleza del problema y las características de las cadenas de entrada, como su simetría o asimetría, afectan el rendimiento de los algoritmos. El análisis de estos dos enfoques no solo proporciona una visión clara sobre sus diferencias fundamentales, sino también sobre cómo cada uno se comporta al enfrentar el problema bajo distintas condiciones y restricciones.