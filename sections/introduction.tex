Hace años que la computación es una de las mejores herramientas para la resolución de problemas en diversas áreas debido a que nos permite procesar grandes volúmenes de datos rápidamente, automatizar tareas y encontrar soluciones optimas que manualmente resultarían imposibles de realizar. Dentro de las ciencias de la computación una de las ramas mas importantes es el \textbf{Análisis y Diseño de Algoritmos}, ya que permite desarrollar diferentes formas para resolver problemas complejos y así, encontrar el método mas eficiente para abordar distintas problemáticas.\\

Este trabajo se enfoca en comparar dos de los principales paradigmas para la creación de algoritmos: \textbf{Fuerza Bruta} y \textbf{Programación Dinámica}, el primero de estos consiste en resolver un problema evaluando todas las posibles soluciones de manera exhaustiva, sin intentar optimizar el proceso de búsqueda. Es una estrategia directa que garantiza encontrar la solución correcta, aunque no sea eficiente en términos de tiempo y recursos. Mientras que, el segundo utiliza un enfoque diseñado para resolver problemas que pueden descomponerse en subproblemas más pequeños y que exhiben las propiedades de solapamiento de estos para llegar a una solución del problema general.\\

Dentro de este reporte será planteado el problema de la \textbf{Distancia Mínima de Edición} como caso de estudio en el que aplicar los enfoques ya mencionados. Este problema consiste en determinar el número mínimo de operaciones necesarias para transformar una cadena A en otra cadena B, mediante las funciones insertar un carácter, eliminar un carácter y reemplazar un carácter. Además, para este experimento se incorpora adicionalmente la operación de transposición, la cual corresponde al intercambio posicional de 2 caracteres adyacentes para igualar cadenas de caracteres.\\

Para lograr un buen experimento sobre la diferencia entre los enfoques utilizados para abordar este problema, junto a evidenciar como el Diseño y Análisis de algoritmos afecta a la obtención e identificación de una solución optima es relevante estudiar la naturaleza del problema, las características de las cadenas de entrada, como su simetría o asimetría afectan el rendimiento de los algoritmos, su complejidad temporal y espacial, etc. Todo esto, a través del estudio de sus respectivos tiempos de ejecución y la calidad de las respuestas proporcionadas por los algoritmo para no solo tener una visión clara sobre sus diferencias fundamentales, sino también sobre cómo cada uno se comporta al enfrentar el problema bajo distintas condiciones y restricciones.\\