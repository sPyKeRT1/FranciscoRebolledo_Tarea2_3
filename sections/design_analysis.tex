El problema de la \textbf{Distancia Mínima de Edición} tiene varias formas de ser abordado, las cuales pueden tener una gran diferencia en la eficiencia de la memoria o los tiempos de ejecución de los algoritmos. Por lo mismo, antes de entrar directamente en el análisis de los algoritmos resulta importante definir el diseño que seguirán los paradigmas para tener en cuenta cuáles son sus principales diferencias y como esto puede llegar a afectar a la solución que dan al problema.\\

En general, estos enfoques buscaran abordar el problema una letra a la vez pero se separan de tal forma que, por \textbf{Fuerza Bruta} el algoritmo deberá calcular y almacenar cada una de las 4 operaciones fundamentales (insertar, borrar, reemplazar y transponer) por cada letra, creando un crecimiento exponencial al buscar cada combinación probable, mientras que el algoritmo con enfoque de \textbf{Programación Dinámica} identificará y almacenará los costos de edición de cadenas mas cortas dentro de la cadena principal para armar en base a estas el costo de la cadena principal.\\

Además, también resulta importante analizar la naturaleza del problema, pues el hecho de que cada operación tenga \textbf{costos variables} en base al o los caracteres involucrados agrega una capa de complejidad a los algoritmos que deben trabajar para decidir qué operación es la más adecuada en cada caso, lo que puede aumentar el tiempo de ejecución de los mismos dadas las operaciones necesarias para el procedimiento. Por lo mismo, agregar la \textbf{operación de transposición} aumenta las posibilidades de los caminos a tomar y variables que tener en cuenta, lo que incrementa aun más la complejidad del problema pero especialmente representa una carga en el enfoque de Fuerza Bruta al considerar exhaustivamente todas las formas de modificar la cadena.\\
\newpage