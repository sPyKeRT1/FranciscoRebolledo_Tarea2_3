El problema de la \textbf{Distancia Mínima de Edición} tiene varias formas de ser abordado, las cuales pueden tener una gran diferencia en la eficiencia de la memoria o los tiempos de ejecución de los algoritmos, por lo mismo, aquí se presentan dos aproximaciones a los algoritmos que dan solución a dicha problemática, en especifico para los paradigmas \textbf{Fuerza Bruta} y \textbf{Programación Dinámica} junto a su respectivo análisis teórico de la complejidad y cómo su estructura afecta a la eficiencia del mismo. Por lo mismo, antes de entrar directamente en el análisis de los algoritmos resulta importante definir cada paradigma para tener en cuenta cuáles son sus principales diferencias y como esto puede llegar a afectar a la solución que dan al problema.\\

El enfoque de \textbf{Fuerza Bruta} consiste en resolver un problema evaluando todas las posibles soluciones de manera exhaustiva, sin intentar optimizar el proceso de búsqueda. Es una estrategia directa que garantiza encontrar la solución correcta, aunque no sea eficiente en términos de tiempo y recursos. Mientras que, la \textbf{Programación Dinámica} utiliza un enfoque diseñado para resolver problemas que pueden descomponerse en subproblemas más pequeños y que exhiben las propiedades de solapamiento de estos para llegar a una solución del problema general.\\

Estos enfoques buscaran abordar el problema de la \textbf{Distancia Mínima de Edición} una letra a la vez pero se separan de tal forma que, por fuerza bruta el algoritmo deberá calcular y almacenar cada una de las 4 operaciones fundamentales (insertar, borrar, reemplazar y transponer) por cada letra, creando un crecimiento exponencial al buscar cada combinación probable, mientras que el algoritmo con enfoque de programación dinámica identificará y almacenará los costos de edición de cadenas mas cortas dentro de la cadena principal para armar en base a estas el costo de la cadena principal.\\

Por último, también resulta importante analizar la naturaleza del problema, pues el hecho de que cada operación tenga \textbf{costos variables} en base al o los caracteres involucrados agrega una capa de complejidad a los algoritmos que deben trabajar para decidir qué operación es la más adecuada en cada caso, lo que puede aumentar el tiempo de ejecución de los mismos dadas las operaciones necesarias para el procedimiento. Por lo mismo, agregar la \textbf{operación de transposición} aumenta las posibilidades de los caminos a tomar y variables que tener en cuenta, lo que incrementa aun más la complejidad del problema pero especialmente representa una carga en el enfoque de Fuerza Bruta al considerar exhaustivamente todas las formas de modificar la cadena.\\