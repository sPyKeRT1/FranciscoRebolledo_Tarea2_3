Para verificar y probar correctamente la funcionalidad de los algoritmos ya sea de Fuerza Bruta o Programación Dinámica aparte de usar los casos aleatorios con crecimiento en el número de caracteres es importante obtener resultados con casos que posean ciertas características distintivas o limites respecto a los demás, es por ello que los 5 principales serán casos con cadenas de caracteres vacíos, cadenas de caracteres repetidos, cadenas simétricas, cadenas asimétricas y donde las matrices tienen un mismo valor para todas las operaciones, casos los cuales tienen al rededor de 5 caracteres por cadena para que esta variable no afecte en los resultados. (Estos datasets pueden ser encontrados en sus respectivas carpetas dentro del repositorio de Github o como imágenes en el apéndice de este reporte)\\

1. \underline{Caso con cadenas vacías:} En este \hyperref[fig:imagen1]{caso} se deben testear cadenas sin ningún carácter para ver el funcionamiento correcto de los algoritmos y ver cual posee mayor efectividad al lidiar con tal problemática.\\

2. \underline{Caso con caracteres repetidas:} La característica de este \hyperref[fig:imagen2]{caso} es que las cadenas son repeticiones de un mismo carácter para ver si la repetición de un subproblema hace al algoritmo más eficiente.\\

3. \underline{Caso con cadenas simétricas:} La intención de este \hyperref[fig:imagen3]{caso} es ver si ayuda de alguna forma a los algoritmos que las cadenas posean igual cantidad de caracteres y compararlo con las cadenas asimétricas.\\

4. \underline{Caso con cadenas asimétricas:} Este \hyperref[fig:imagen4]{caso} va de la mano con el anterior y sirve para verificar si las cadenas de distinto largo tienen algún impacto en los tiempos de ejecución y la respuesta de los algoritmos.\\

5. \underline{Caso con matrices con valores iguales:} Por último \hyperref[fig:imagen5]{caso}, resulta interesante verificar el efecto de matrices iguales o planas en sus valores y como esto puede afectar en la calidad de sus respuestas.\\
