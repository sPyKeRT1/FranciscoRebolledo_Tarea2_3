En base a los resultados obtenidos, se puede concluir que el enfoque de Programación Dinámica ofrece una solución mucho más eficiente al problema de la Distancia Mínima de Edición que el enfoque de Fuerza Bruta, especialmente cuando se trata de cadenas de caracteres largas. La Fuerza Bruta presenta un crecimiento exponencial en el tiempo de ejecución a medida que se incrementan los caracteres, lo que la convierte en una opción inviable para entradas de mayor tamaño debido a su complejidad. En cambio, la Programación Dinámica optimiza este proceso al dividir el problema en subproblemas más pequeños y reutilizar los resultados previamente calculados, lo que reduce significativamente los tiempos de ejecución y hace que el algoritmo sea escalable.\\

Además, al considerar cadenas simétricas y asimétricas, se observó que la Fuerza Bruta se ve gravemente afectada en el caso de cadenas asimétricas, donde la cantidad de combinaciones posibles de operaciones es mucho mayor. En comparación, el algoritmo de Programación Dinámica se comporta de manera mucho más eficiente independientemente de la simetría de las cadenas, demostrando la robustez de este enfoque. En resumen, la Programación Dinámica es claramente la mejor opción para resolver el problema de la Distancia Mínima de Edición en términos de tiempo y eficiencia, especialmente cuando se manejan cadenas de texto grandes o complejas. En general, se verifica la hipótesis preliminar respecto a los enfoques y como estos pueden cambiar la resolucion de un problema.\\

Una posible mejora sería optimizar el uso de memoria mediante la reducción del espacio de almacenamiento necesario. En lugar de utilizar una matriz completa para almacenar los resultados de todos los subproblemas, se podría emplear una estructura de memoria más compacta, como una matriz unidimensional o una técnica de compresión de los resultados intermedios. De todos modos, independiente de los resultados obtenidos se evidencio que existe una falencia en la comprobación de la complejidad espacial, es decir, el uso de memoria lo cual debe ser mejorado y abarcado de mejor forma en posteriores investigaciones.